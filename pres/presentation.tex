\documentclass{beamer}

\usepackage[utf8]{inputenc}
\usetheme{Warsaw}

\title{Embedded Systems - Bachelor Project}
\author{Moritz Herzog, Philipp Lersch}
\date{\today}

\addtobeamertemplate{navigation symbols}{}{%
    \usebeamerfont{footline}%
    \usebeamercolor[fg]{footline}%
    \hspace{1em}%
    \insertframenumber/\inserttotalframenumber
}

\AtBeginSection[]
{
  \begin{frame}
    \frametitle{Table of Contents}
    \tableofcontents[currentsection]
  \end{frame}
}

\begin{document}
\maketitle


\section{OpenCL}
\begin{frame}
    \frametitle{OpenCL Structure}
    
\end{frame}
\subsection{Hardware Model}
\begin{frame}
    \frametitle{Hardware Model}
\end{frame}
\subsection{Software Model}
\begin{frame}
    \frametitle{Software Model}
\end{frame}

\section{Kernels}
\begin{frame} %%Eine Folie
  \frametitle{List of Kernels} %%Folientitel
  Already there:
  \begin{itemize}
   \item Battery
   \item Speed
   \item Acceleration
   \begin{itemize}
    \item X-Achsis
    \item Y-Achsis
    \item Tangential
   \end{itemize}
   \item Tempreture
  \end{itemize}

  %\begin{Definition} %%Definition
  %  Eine Definition
  %\end{Definition}
\end{frame}
\begin{frame}
    \frametitle{List of Kernels}
    Added Kernels:
    \begin{itemize}
     \item Tractioncontrol
     \item Accident Detection
     \item Anti-Lock Braking System
    \end{itemize}
\end{frame}
\subsection{Tractioncontrol}
\begin{frame}
    \frametitle{Accident Detection}
    By testing for accelerationspikes or accelerationvalues, we are able to detect accidents and unexpected movement.\\
    \begin{itemize}
     \item Check for values above or below given thresholds which should be unachieveable under normal circumstances.\\
     $\Rightarrow$ If found, we crashed.
     \pause
     \item In addition we can look for spikes in the last n-datapoints of the acceration and determine light crashes.\\
     $\Rightarrow$ If found, we crashed.
     \pause
    \end{itemize}
\end{frame}
\subsection{Anti Blocking System}
\begin{frame}
    \frametitle{Anti Blocking System}
    By comparing the fronts average and back wheels speeds we can determine if the rear wheels are turning to slow or are locking up. 
    \begin{itemize}
     \item Check if the rear wheels are spinning much slower (threshold is around 40\%) than the front wheels.\\
     $\Rightarrow$ If this is true for one or both wheels we are breaking to hard.\\
     $\Rightarrow$ Signal traction loss.
     \pause
     \item We could also try to regain traction by spinning the motor a little bit faster until we regain traction and slow down again.\\
     $\Rightarrow$ Hard to implement without proper or near realtime capeabliltys and no finer specified wheel control. Breaking has to be implemented by slowing down the build in motor. \\
     \pause
    \end{itemize}
\end{frame}
\subsection{Antriebsschlupfregelung}
\begin{frame}
    By comparing the fronts average and back wheels speeds we can determine if the rear wheels are turning to fast. 
    \begin{itemize}
     \item Check if the rear wheels are spinning much faster (threshold is around 40\%) than the front wheels.\\
     $\Rightarrow$ If this is true for one or both wheels we are accelerating to hard.\\
     $\Rightarrow$ Signal traction loss.
     \pause
     \item We could also try to regain traction by spinning the motor a little bit slower until we regain traction and slow down again.\\
     $\Rightarrow$ Hard to implement without proper or near realtime capeablilties and no finer specifieable wheel control like a differential lock. Could be implemented by flattening the acceleration curve for the motor. \\
     \pause
    \end{itemize}
\end{frame}
\subsection{Tractioncontrol}
\begin{frame}
    \frametitle{Tractioncontrol}
    By comparing the fronts average and back wheels speeds we can determine if the rear wheels have lost to much traction.
    \begin{itemize}
     \item Check if the rear wheels are spinning outside of their thresholds.\\
     $\Rightarrow$ If one or both wheels are to slow we are breaking to hard.\\
     $\Rightarrow$ Signal traction loss.\\
     $\Rightarrow$ If one or both wheels are to fast we are accelerating to hard.\\
     $\Rightarrow$ Signal traction loss.
     \pause
     \item Without the needed capeablilties we are only cabealbe to signal the loss of traction.
     \pause
    \end{itemize}
\end{frame}
\subsection{Range}
\begin{frame}
 %TODO To be filled
\end{frame}
\end{document}
